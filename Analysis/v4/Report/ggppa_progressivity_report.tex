
\documentclass[11pt]{article}

\usepackage[utf8]{inputenc}
\usepackage[T1]{fontenc}
\usepackage[margin=1in]{geometry}
\usepackage{booktabs}
\usepackage{longtable}
\usepackage{amsmath, amssymb}
\usepackage{graphicx}
\usepackage{hyperref}
\usepackage{setspace}
\usepackage{caption}
\usepackage{array}

\hypersetup{
  colorlinks=true,
  linkcolor=blue,
  urlcolor=blue,
  citecolor=blue
}

\title{Evaluating the Progressivity of Canada's Federal Fuel Charge (GGPPA Part I) and the Canada Carbon Rebate (CCR), 2019--2025\\
\large Ontario, Manitoba, Saskatchewan, Alberta}
\author{Replication package generated by Python incidence model}
\date{\today}

\begin{document}
\maketitle
\vspace{-0.5em}

\begin{abstract}
This report documents a reduced-form incidence model for Canada's federal fuel charge under Part I of the \textit{Greenhouse Gas Pollution Pricing Act} (GGPPA) and the associated household rebate (Climate Action Incentive / Canada Carbon Rebate, CAI/CCR). The model produces province--income-quintile--year estimates of (i) \emph{gross} household costs of the fuel charge and (ii) \emph{net} household costs after subtracting CCR schedules, together with effective tax rates (ETRs) defined relative to total household expenditure. The scope is restricted to Ontario, Manitoba, Saskatchewan and Alberta from 2019 (policy start) through 2025 (policy wind-down convention). All intermediate quantities and component taxes are saved to the main CSV output to maximize transparency and auditability.
\end{abstract}

\section{Policy background and scope}
Part I of the GGPPA establishes a federal \emph{fuel charge} applied to a broad set of fossil fuels in jurisdictions that do not meet the federal benchmark. The charge is legislated in Schedule~2 of the Act and administered by the Canada Revenue Agency (CRA). Rates for key fuels in backstop provinces from April~2019 to March~2025 are published by the CRA in \textit{Fuel charge rates}.

Household proceeds are returned primarily through a rebate administered through the personal income tax system (originally the Climate Action Incentive; renamed the Canada Carbon Rebate in 2024). Department of Finance backgrounders publish annual rebate schedules by province and household composition and describe the rural supplement. Environment and Climate Change Canada (ECCC) annual reporting on the administration of the GGPPA documents, among other items, the schedule of fuel charge rates and the wind-down of the consumer fuel charge effective April~1,~2025.

This report focuses exclusively on Ontario, Manitoba, Saskatchewan, and Alberta; years 2019--2025; and the consumer fuel charge (Part I). The output-based pricing system (Part II) is outside scope except insofar as it informs external cross-validation benchmarks.

\section{Data}

\subsection{User-provided micro-aggregation inputs (Statistics Canada)}
Three Statistics Canada tables (provided as \texttt{.xlsx} extracts) are used:

\begin{itemize}
  \item \textbf{Household spending by category}: Table 11-10-0223-01 (Survey of Household Spending) provides average annual expenditures by province and before-tax income quintile for:
  \begin{align}
    E^{gas}_{pqt} &\equiv \text{``Gas and other fuels (all vehicles and tools)''},\\
    E^{ng}_{pqt} &\equiv \text{``Natural gas for principal accommodation''},\\
    E^{oil}_{pqt} &\equiv \text{``Other fuel for principal accommodation''},\\
    E^{tot}_{pqt} &\equiv \text{``Total expenditure''}.
  \end{align}
  \item \textbf{Retail energy prices}: Table 18-10-0001-01 provides gasoline and heating-fuel retail prices. The script uses annual means of:
  \begin{align}
    P^{gas}_{pt} &\equiv \text{``Regular unleaded gasoline at self service filling stations''},\\
    P^{oil}_{pt} &\equiv \text{``Household heating fuel''}.
  \end{align}
  \item \textbf{Household counts}: Table 36-10-0101-01 provides the number of households by province and income quintile, used as weights $w_{pqt}$.
\end{itemize}

\paragraph{Interpolation.} The expenditure extract includes years 2019, 2021, and 2023; the script linearly interpolates missing years within each province$\times$quintile cell. For 2024--2025, expenditures and household counts are held at their 2023 interpolated level in the absence of additional microdata. This maintained assumption affects the level of later-year incidence estimates and is documented in the code and the data dictionary.

\subsection{Statutory rates and rebate schedules}
\begin{itemize}
  \item \textbf{Fuel charge rates.} Statutory rates for gasoline, light fuel oil, and marketable natural gas are taken from a user-provided extract \texttt{Tax Rate Items\_v2.xlsx}. The underlying primary source is the CRA \textit{Fuel charge rates} publication, which reports historical rates (April~2019 to March~2025) by fuel and fuel-charge year.
  \item \textbf{CCR/CAI schedules.} Annual base rebate schedules for 2019--2024 are hard-coded in the script based on Department of Finance backgrounders for the corresponding benefit years. The rural supplement rate is parameterized as $10\%$ through 2023 and $20\%$ in 2024 (consistent with the announced policy change for 2024--25). The script sets rebate schedules to zero for 2025 to match its Part I wind-down convention.
  \item \textbf{Carbon price schedule.} For transparency and for indirect-effect calibration, the script stores a carbon price schedule in $\$/\mathrm{tCO_2e}$ as \texttt{carbon\_price\_per\_tCO2e}. This is used only for interpolation of the indirect wedge; direct statutory rates are taken from the CRA schedule and are not extrapolated.
\end{itemize}

\section{Methodology}

\subsection{Imputing fuel quantities from expenditures}
The model converts expenditures into physical quantities using annual average retail prices.

\paragraph{Gasoline and heating oil.} For gasoline and heating oil, quantities are imputed by dividing expenditure by the annual average retail price:
\begin{align}
  \widehat{Q}^{gas}_{pqt} &= \frac{E^{gas}_{pqt}}{P^{gas}_{pt}} \quad (\text{litres}), \\
  \widehat{Q}^{oil}_{pqt} &= \frac{E^{oil}_{pqt}}{P^{oil}_{pt}} \quad (\text{litres}).
\end{align}

\paragraph{Natural gas (reduced form).} The SHS extract provides natural-gas expenditures but not physical quantity. The script uses a reduced-form conversion:
\begin{align}
  \widehat{Q}^{ng}_{pqt} &= \phi_p \cdot E^{ng}_{pqt} \quad (\text{GJ}),
\end{align}
where $\phi_p$ is a province-specific ``GJ per dollar'' factor. The coefficients are calculated from StatCan Table 25-10-0060-01:
\[
\phi_{ON}=0.0720192545,\;\phi_{MB}=0.1386298932,\;\phi_{SK}=0.0878257722,\;\phi_{AB}=0.0858479021.
\]
Interpretation: $\phi_p \approx 1/\overline{P}^{ng}_p$, the inverse of an assumed average residential end-user natural gas price in \$/GJ. This reduced-form step is necessary because the SHS table provides expenditures but not natural gas quantities. The script also converts the statutory natural gas rate from \$/m$^3$ to \$/GJ using a constant conversion factor:
\begin{align}
  r^{ng,GJ}_{pt} &= r^{ng,m^3}_{pt}\cdot \kappa^{m^3/GJ}, \qquad \kappa^{m^3/GJ}=26.853.
\end{align}

\subsection{Direct fuel charge incidence}
Let $r^{gas}_{pt}$ denote the statutory gasoline rate (\$/L), $r^{oil}_{pt}$ the light fuel oil rate (\$/L), and $r^{ng,GJ}_{pt}$ the natural gas rate (\$/GJ). Direct fuel charge paid is:
\begin{align}
  T^{gas}_{pqt} &= \widehat{Q}^{gas}_{pqt}\cdot r^{gas}_{pt},\\
  T^{oil}_{pqt} &= \widehat{Q}^{oil}_{pqt}\cdot r^{oil}_{pt},\\
  T^{ng}_{pqt}  &= \widehat{Q}^{ng}_{pqt}\cdot r^{ng,GJ}_{pt},\\
  T^{dir}_{pqt} &= T^{gas}_{pqt}+T^{oil}_{pqt}+T^{ng}_{pqt}.
\end{align}
The output CSV includes each component ($T^{gas}_{pqt}$, $T^{oil}_{pqt}$, $T^{ng}_{pqt}$) and intermediate quantities for transparency.

\subsection{Indirect effects: reduced-form wedge calibrated to published anchors}
The model represents indirect effects (price impacts on other goods and services) using a multiplicative wedge $\kappa_{pt}$ applied to the direct burden:
\begin{align}
  T^{gross}_{pqt} &= T^{dir}_{pqt}\cdot (1+\kappa_{pt}), \\
  T^{ind}_{pqt} &= T^{gross}_{pqt}-T^{dir}_{pqt}.
\end{align}

Calibration proceeds in two steps. First compute the household-weighted provincial average direct burden:
\begin{align}
  \overline{T}^{dir}_{pt} = \frac{\sum_q w_{pqt}T^{dir}_{pqt}}{\sum_q w_{pqt}}.
\end{align}
Given an external ``average cost impact per household'' anchor $A_{pt}$, the implied wedge is:
\begin{align}
  \kappa_{pt} = \frac{A_{pt}}{\overline{T}^{dir}_{pt}} - 1.
\end{align}
Second, the script fits a province-specific OLS line in carbon-price space using anchor years 2021, 2022, and 2024:
\begin{align}
  \kappa_{pt} = \alpha_p + \beta_p P_t.
\end{align}
Anchor years are imposed exactly at their implied values, while non-anchor years use the fitted value (floored at zero). In 2025, the script sets direct statutory rates to zero and imposes $\kappa_{pt}=0$.

\subsection{CCR/CAI: net incidence under alternative household types}
Let $R^{type}_{pt}$ denote the annual CCR schedule amount for household type \texttt{type} $\in \{$single adult, couple, family of four$\}$. Net cost is:
\begin{align}
  T^{net,type}_{pqt} = T^{gross}_{pqt} - R^{type}_{pt}.
\end{align}
A rural supplement is implemented as:
\begin{align}
  R^{type,rural}_{pt} = R^{type}_{pt}\cdot(1+\tau_t),
\end{align}
with $\tau_t$ the rural top-up rate.

\subsection{Effective tax rates (ETRs)}
For any incidence measure $X_{pqt}\in\{T^{dir}_{pqt},T^{gross}_{pqt},T^{net,type}_{pqt}\}$, the effective tax rate is:
\begin{align}
  \text{ETR}(X)_{pqt} = \frac{X_{pqt}}{E^{tot}_{pqt}}.
\end{align}

\section{Results}

\subsection{Illustrative ETR patterns in 2024}
Figure~\ref{fig:etr_gross_2024} plots the gross-cost ETR by quintile in 2024. Figure~\ref{fig:etr_net_2024} shows the net ETR after subtracting the base CCR for a family of four. Negative net ETR values indicate net gains (rebate exceeds gross cost).

\begin{figure}[htbp]\centering
  \includegraphics[width=0.88\textwidth]{fig_etr_gross_2024.pdf}
  \caption{Gross-cost effective tax rates in 2024 (direct fuel charge plus indirect wedge), by income quintile.}
  \label{fig:etr_gross_2024}
\end{figure}

\begin{figure}[htbp]\centering
  \includegraphics[width=0.88\textwidth]{fig_etr_net_family4_base_2024.pdf}
  \caption{Net effective tax rates in 2024 for a family of four (gross cost minus base CCR), by income quintile.}
  \label{fig:etr_net_2024}
\end{figure}

\subsection{Time trend in provincial average gross costs}
Figure~\ref{fig:trend} displays the household-weighted provincial average gross cost over 2019--2025. The sharp drop in 2025 reflects the script convention of setting fuel charge rates to zero.

\begin{figure}[htbp]\centering
  \includegraphics[width=0.88\textwidth]{fig_gross_cost_avg_trend.pdf}
  \caption{Household-weighted provincial average gross cost over time.}
  \label{fig:trend}
\end{figure}

\subsection{Selected incidence levels (2021 and 2024)}
Table~\ref{tab:sel_results} reports gross costs and net costs (family of four, base CCR) by quintile in 2021 and 2024 (ETRs in percent). This table is generated directly from the output CSV.

\begin{longtable}{lrlrrrr}
\caption{Incidence measures by province and income quintile (selected years). ETRs are percentages of total expenditure.} \label{tab:sel_results} \\
\toprule
Province & Year & Quintile & gross\_cost & net\_family4\_base & etr\_gross\_cost & etr\_net\_family4\_base \\
\midrule
\endfirsthead
\caption[]{Incidence measures by province and income quintile (selected years). ETRs are percentages of total expenditure.} \\
\toprule
Province & Year & Quintile & gross\_cost & net\_family4\_base & etr\_gross\_cost & etr\_net\_family4\_base \\
\midrule
\endhead
\midrule
\multicolumn{7}{r}{Continued on next page} \\
\midrule
\endfoot
\bottomrule
\endlastfoot
Alberta & 2021 & Fourth & 682 & -299 & 0.550000 & -0.240000 \\
Alberta & 2021 & Highest & 761 & -220 & 0.390000 & -0.110000 \\
Alberta & 2021 & Lowest & 266 & -715 & 0.620000 & -1.660000 \\
Alberta & 2021 & Second & 480 & -501 & 0.700000 & -0.730000 \\
Alberta & 2021 & Third & 508 & -473 & 0.580000 & -0.540000 \\
Alberta & 2024 & Fourth & 1045 & -755 & 0.810000 & -0.580000 \\
Alberta & 2024 & Highest & 1366 & -434 & 0.570000 & -0.180000 \\
Alberta & 2024 & Lowest & 538 & -1262 & 1.120000 & -2.620000 \\
Alberta & 2024 & Second & 991 & -809 & 1.230000 & -1.010000 \\
Alberta & 2024 & Third & 955 & -845 & 0.900000 & -0.800000 \\
Manitoba & 2021 & Fourth & 542 & -178 & 0.540000 & -0.180000 \\
Manitoba & 2021 & Highest & 777 & 57 & 0.470000 & 0.030000 \\
Manitoba & 2021 & Lowest & 222 & -498 & 0.570000 & -1.280000 \\
Manitoba & 2021 & Second & 306 & -414 & 0.550000 & -0.750000 \\
Manitoba & 2021 & Third & 505 & -215 & 0.630000 & -0.270000 \\
Manitoba & 2024 & Fourth & 1104 & -96 & 1.010000 & -0.090000 \\
Manitoba & 2024 & Highest & 1427 & 227 & 0.780000 & 0.120000 \\
Manitoba & 2024 & Lowest & 341 & -859 & 0.900000 & -2.260000 \\
Manitoba & 2024 & Second & 540 & -660 & 0.900000 & -1.090000 \\
Manitoba & 2024 & Third & 763 & -437 & 0.980000 & -0.560000 \\
Ontario & 2021 & Fourth & 603 & 3 & 0.520000 & 0.000000 \\
Ontario & 2021 & Highest & 641 & 41 & 0.330000 & 0.020000 \\
Ontario & 2021 & Lowest & 260 & -340 & 0.640000 & -0.840000 \\
Ontario & 2021 & Second & 295 & -305 & 0.530000 & -0.550000 \\
Ontario & 2021 & Third & 380 & -220 & 0.470000 & -0.270000 \\
Ontario & 2024 & Fourth & 1059 & -61 & 0.820000 & -0.050000 \\
Ontario & 2024 & Highest & 1275 & 155 & 0.520000 & 0.060000 \\
Ontario & 2024 & Lowest & 437 & -683 & 0.980000 & -1.530000 \\
Ontario & 2024 & Second & 789 & -331 & 1.070000 & -0.450000 \\
Ontario & 2024 & Third & 800 & -320 & 0.840000 & -0.340000 \\
Saskatchewan & 2021 & Fourth & 824 & -176 & 0.760000 & -0.160000 \\
Saskatchewan & 2021 & Highest & 1007 & 7 & 0.580000 & 0.000000 \\
Saskatchewan & 2021 & Lowest & 451 & -549 & 1.080000 & -1.320000 \\
Saskatchewan & 2021 & Second & 512 & -488 & 0.870000 & -0.830000 \\
Saskatchewan & 2021 & Third & 672 & -328 & 0.940000 & -0.460000 \\
Saskatchewan & 2024 & Fourth & 1307 & -197 & 1.160000 & -0.170000 \\
Saskatchewan & 2024 & Highest & 1770 & 266 & 1.000000 & 0.150000 \\
Saskatchewan & 2024 & Lowest & 503 & -1001 & 1.270000 & -2.520000 \\
Saskatchewan & 2024 & Second & 856 & -648 & 1.400000 & -1.060000 \\
Saskatchewan & 2024 & Third & 1012 & -492 & 1.310000 & -0.640000 \\
\end{longtable}


\section{Cross-validation and integrity checks}
This section documents checks against published sources and built-in diagnostics. The aim is to verify schedules and magnitudes; definitions can differ across sources (calendar vs benefit years, included channels, treatment of GST, and inclusion of broader economic effects).

\subsection{Fuel charge rates (CRA)}
The model's statutory fuel charge rates are sourced from \texttt{Tax Rate Items\_v2.xlsx}, which corresponds to the CRA's published historical fuel charge rates (April 2019 to March 2025). The CRA publication reports gasoline rates of 0.0442, 0.0663, 0.0884, 0.1105, 0.1431, and 0.1761 \$/L for successive fuel-charge years through 2024--25, and marketable natural gas rates of 0.0391 to 0.1525 \$/m$^3$ over the same period. These values are consistent with the units used in this model (direct taxes computed as rate times imputed litres or GJ). 

\subsection{CCR/CAI schedules (Department of Finance and CRA)}
The script's CCR/CAI schedule parameters match Department of Finance backgrounders for each relevant benefit year: 2019, 2020, 2021, 2022--23, 2023--24, and 2024--25. CRA program pages on the Canada Carbon Rebate summarize quarterly payment amounts for previous base years and provide an additional check that the annual totals are in the correct range.

\subsection{Indirect wedge calibration (government cost-impact anchors)}
The file \texttt{ggppa\_finance\_crosswalk\_v4.csv} provides a diagnostic comparison between the model's household-weighted provincial average gross cost and the anchor values used for calibration in 2021, 2022, and 2024. Table~\ref{tab:finance_crosswalk} reproduces this crosswalk.

\begin{table}
\caption{Calibration diagnostic: model-implied provincial average costs versus anchors used for calibration (see script dictionary FINANCE\_COST\_IMPACT).}
\label{tab:finance_crosswalk}
\begin{tabular}{lrrrrr}
\toprule
Province & Year & Direct\_model\_avg & Gross\_model\_avg & Finance\_cost\_impact & Gross\_minus\_finance \\
\midrule
Alberta & 2021 & 370.661989 & 598.000000 & 598.000000 & -0.000000 \\
Alberta & 2022 & 503.143296 & 700.000000 & 700.000000 & 0.000000 \\
Alberta & 2024 & 946.499849 & 1056.000000 & 1056.000000 & 0.000000 \\
Manitoba & 2021 & 294.127095 & 462.000000 & 462.000000 & 0.000000 \\
Manitoba & 2022 & 338.154679 & 559.000000 & 559.000000 & 0.000000 \\
Manitoba & 2024 & 643.640426 & 828.000000 & 828.000000 & 0.000000 \\
Ontario & 2021 & 267.464018 & 439.000000 & 439.000000 & 0.000000 \\
Ontario & 2022 & 318.358015 & 578.000000 & 578.000000 & 0.000000 \\
Ontario & 2024 & 566.797227 & 869.000000 & 869.000000 & 0.000000 \\
Saskatchewan & 2021 & 363.083032 & 720.000000 & 720.000000 & -0.000000 \\
Saskatchewan & 2022 & 440.575700 & 734.000000 & 734.000000 & -0.000000 \\
Saskatchewan & 2024 & 795.748321 & 1156.000000 & 1156.000000 & 0.000000 \\
\bottomrule
\end{tabular}
\end{table}


\subsection{Independent distributional analysis (PBO)}
The Parliamentary Budget Officer (PBO) provides a distributional analysis of federal carbon pricing for Ontario, Manitoba, Saskatchewan, and Alberta, reporting gross costs and net costs by income quintile and emphasizing that distributional results depend on whether broader economic channels are included. While the PBO framework is not identical to this report (it incorporates additional channels and reports fiscal-year results under a different forward policy scenario), it is a useful external benchmark. In particular, the PBO documents that gross household carbon costs vary by province and that net costs (after rebates) can change sign across the income distribution. 

For completeness, Table~\ref{tab:kakwani} reports Kakwani indices produced by the script for selected measures (direct tax, gross cost, and net cost for a single adult under the base schedule), using total expenditure as the ranking variable. These indices summarize whether a burden is progressive or regressive relative to expenditure.

\begin{longtable}{lrlrrrr}
\caption{Kakwani indices and related inequality statistics (ranking variable: total expenditure).} \label{tab:kakwani} \\
\toprule
Province & Year & Measure & Kakwani & Gini\_expenditure & Concentration & Total\_net\_revenue \\
\midrule
\endfirsthead
\caption[]{Kakwani indices and related inequality statistics (ranking variable: total expenditure).} \\
\toprule
Province & Year & Measure & Kakwani & Gini\_expenditure & Concentration & Total\_net\_revenue \\
\midrule
\endhead
\midrule
\multicolumn{7}{r}{Continued on next page} \\
\midrule
\endfoot
\bottomrule
\endlastfoot
Alberta & 2019 & direct\_tax & NaN & 0.277000 & NaN & 0.000000 \\
Alberta & 2019 & gross\_cost & NaN & 0.277000 & NaN & 0.000000 \\
Alberta & 2019 & net\_single\_base & NaN & 0.277000 & NaN & 0.000000 \\
Alberta & 2020 & direct\_tax & -0.127000 & 0.275000 & 0.148000 & 484402246.465353 \\
Alberta & 2020 & gross\_cost & -0.127000 & 0.275000 & 0.148000 & 1013219994.386693 \\
Alberta & 2020 & net\_single\_base & 0.280000 & 0.275000 & 0.555000 & 244527234.386693 \\
Alberta & 2021 & direct\_tax & -0.105000 & 0.260000 & 0.155000 & 535415576.169204 \\
Alberta & 2021 & gross\_cost & -0.105000 & 0.260000 & 0.155000 & 1037167014.000000 \\
Alberta & 2021 & net\_single\_base & 0.533000 & 0.260000 & 0.793000 & 187314444.000000 \\
Alberta & 2022 & direct\_tax & -0.125000 & 0.276000 & 0.151000 & 737219306.651247 \\
Alberta & 2022 & gross\_cost & -0.125000 & 0.276000 & 0.151000 & 1314136963.074728 \\
Alberta & 2022 & net\_single\_base & 0.220000 & 0.276000 & 0.496000 & 362654277.074729 \\
Alberta & 2023 & direct\_tax & -0.142000 & 0.278000 & 0.136000 & 1246016829.838057 \\
Alberta & 2023 & gross\_cost & -0.142000 & 0.278000 & 0.136000 & 1932210577.291688 \\
Alberta & 2023 & net\_single\_base & 0.188000 & 0.278000 & 0.467000 & 513942357.291688 \\
Alberta & 2024 & direct\_tax & -0.140000 & 0.284000 & 0.144000 & 1533196983.980337 \\
Alberta & 2024 & gross\_cost & -0.140000 & 0.284000 & 0.144000 & 2022073152.000000 \\
Alberta & 2024 & net\_single\_base & 0.573000 & 0.284000 & 0.857000 & 298715352.000000 \\
Alberta & 2025 & direct\_tax & NaN & 0.284000 & NaN & 0.000000 \\
Alberta & 2025 & gross\_cost & NaN & 0.284000 & NaN & 0.000000 \\
Alberta & 2025 & net\_single\_base & NaN & 0.284000 & NaN & 0.000000 \\
Manitoba & 2019 & direct\_tax & -0.045000 & 0.292000 & 0.247000 & 72557188.572848 \\
Manitoba & 2019 & gross\_cost & -0.045000 & 0.292000 & 0.247000 & 169538725.301776 \\
Manitoba & 2019 & net\_single\_base & 0.210000 & 0.292000 & 0.502000 & 77586915.301776 \\
Manitoba & 2020 & direct\_tax & -0.042000 & 0.291000 & 0.248000 & 118072345.348980 \\
Manitoba & 2020 & gross\_cost & -0.042000 & 0.291000 & 0.248000 & 263051373.881658 \\
Manitoba & 2020 & net\_single\_base & 0.174000 & 0.291000 & 0.465000 & 130387224.881658 \\
Manitoba & 2021 & direct\_tax & -0.038000 & 0.275000 & 0.237000 & 119196837.865487 \\
Manitoba & 2021 & gross\_cost & -0.038000 & 0.275000 & 0.237000 & 252595266.000000 \\
Manitoba & 2021 & net\_single\_base & 0.676000 & 0.275000 & 0.950000 & 55767786.000000 \\
Manitoba & 2022 & direct\_tax & -0.032000 & 0.290000 & 0.258000 & 133624789.590999 \\
Manitoba & 2022 & gross\_cost & -0.032000 & 0.290000 & 0.258000 & 268639946.799843 \\
Manitoba & 2022 & net\_single\_base & 1.357000 & 0.290000 & 1.647000 & 37189610.799843 \\
Manitoba & 2023 & direct\_tax & -0.028000 & 0.279000 & 0.250000 & 210642320.739382 \\
Manitoba & 2023 & gross\_cost & -0.028000 & 0.279000 & 0.250000 & 389118822.602559 \\
Manitoba & 2023 & net\_single\_base & 0.750000 & 0.279000 & 1.029000 & 88935510.602559 \\
Manitoba & 2024 & direct\_tax & -0.024000 & 0.289000 & 0.265000 & 285349358.600911 \\
Manitoba & 2024 & gross\_cost & -0.024000 & 0.289000 & 0.265000 & 480581964.000000 \\
Manitoba & 2024 & net\_single\_base & 0.593000 & 0.289000 & 0.882000 & 132334164.000000 \\
Manitoba & 2025 & direct\_tax & NaN & 0.289000 & NaN & 0.000000 \\
Manitoba & 2025 & gross\_cost & NaN & 0.289000 & NaN & 0.000000 \\
Manitoba & 2025 & net\_single\_base & NaN & 0.289000 & NaN & 0.000000 \\
Ontario & 2019 & direct\_tax & -0.098000 & 0.312000 & 0.214000 & 812317674.252460 \\
Ontario & 2019 & gross\_cost & -0.098000 & 0.312000 & 0.214000 & 1555275941.747302 \\
Ontario & 2019 & net\_single\_base & 0.131000 & 0.312000 & 0.443000 & 667742679.747303 \\
Ontario & 2020 & direct\_tax & -0.097000 & 0.315000 & 0.218000 & 1291530322.772106 \\
Ontario & 2020 & gross\_cost & -0.097000 & 0.315000 & 0.218000 & 2445634893.887338 \\
Ontario & 2020 & net\_single\_base & 0.104000 & 0.315000 & 0.419000 & 1139306765.887338 \\
Ontario & 2021 & direct\_tax & -0.096000 & 0.323000 & 0.226000 & 1377739829.264906 \\
Ontario & 2021 & gross\_cost & -0.096000 & 0.323000 & 0.226000 & 2579919590.000000 \\
Ontario & 2021 & net\_single\_base & 0.289000 & 0.323000 & 0.611000 & 816876590.000000 \\
Ontario & 2022 & direct\_tax & -0.110000 & 0.336000 & 0.225000 & 1629754369.470576 \\
Ontario & 2022 & gross\_cost & -0.110000 & 0.336000 & 0.225000 & 3017576725.474115 \\
Ontario & 2022 & net\_single\_base & 0.377000 & 0.336000 & 0.713000 & 782011669.474115 \\
Ontario & 2023 & direct\_tax & -0.124000 & 0.342000 & 0.218000 & 2413416937.982651 \\
Ontario & 2023 & gross\_cost & -0.124000 & 0.342000 & 0.218000 & 4392471716.764915 \\
Ontario & 2023 & net\_single\_base & 0.231000 & 0.342000 & 0.573000 & 1381223308.764915 \\
Ontario & 2024 & direct\_tax & -0.125000 & 0.336000 & 0.212000 & 3080052208.685105 \\
Ontario & 2024 & gross\_cost & -0.125000 & 0.336000 & 0.212000 & 5508644878.000000 \\
Ontario & 2024 & net\_single\_base & 0.169000 & 0.336000 & 0.505000 & 1958770158.000000 \\
Ontario & 2025 & direct\_tax & NaN & 0.336000 & NaN & 0.000000 \\
Ontario & 2025 & gross\_cost & NaN & 0.336000 & NaN & 0.000000 \\
Ontario & 2025 & net\_single\_base & NaN & 0.336000 & NaN & 0.000000 \\
Saskatchewan & 2019 & direct\_tax & -0.069000 & 0.283000 & 0.214000 & 88618404.965814 \\
Saskatchewan & 2019 & gross\_cost & -0.069000 & 0.283000 & 0.214000 & 248105321.419431 \\
Saskatchewan & 2019 & net\_single\_base & 0.206000 & 0.283000 & 0.489000 & 102587076.419431 \\
Saskatchewan & 2020 & direct\_tax & -0.084000 & 0.275000 & 0.191000 & 138050529.962763 \\
Saskatchewan & 2020 & gross\_cost & -0.084000 & 0.275000 & 0.191000 & 363210719.507292 \\
Saskatchewan & 2020 & net\_single\_base & 0.119000 & 0.275000 & 0.394000 & 169380554.507292 \\
Saskatchewan & 2021 & direct\_tax & -0.102000 & 0.258000 & 0.156000 & 140082721.822203 \\
Saskatchewan & 2021 & gross\_cost & -0.102000 & 0.258000 & 0.156000 & 344924640.000000 \\
Saskatchewan & 2021 & net\_single\_base & 0.237000 & 0.258000 & 0.495000 & 105393640.000000 \\
Saskatchewan & 2022 & direct\_tax & -0.083000 & 0.260000 & 0.177000 & 169662590.302106 \\
Saskatchewan & 2022 & gross\_cost & -0.083000 & 0.260000 & 0.177000 & 389135859.815847 \\
Saskatchewan & 2022 & net\_single\_base & 0.288000 & 0.260000 & 0.548000 & 121047159.815847 \\
Saskatchewan & 2023 & direct\_tax & -0.066000 & 0.259000 & 0.193000 & 256229530.821538 \\
Saskatchewan & 2023 & gross\_cost & -0.066000 & 0.259000 & 0.193000 & 522843516.256834 \\
Saskatchewan & 2023 & net\_single\_base & 0.269000 & 0.259000 & 0.528000 & 185957916.256834 \\
Saskatchewan & 2024 & direct\_tax & -0.066000 & 0.258000 & 0.192000 & 327342160.717194 \\
Saskatchewan & 2024 & gross\_cost & -0.066000 & 0.258000 & 0.192000 & 585114024.000000 \\
Saskatchewan & 2024 & net\_single\_base & 0.276000 & 0.258000 & 0.534000 & 204486216.000000 \\
Saskatchewan & 2025 & direct\_tax & NaN & 0.258000 & NaN & 0.000000 \\
Saskatchewan & 2025 & gross\_cost & NaN & 0.258000 & NaN & 0.000000 \\
Saskatchewan & 2025 & net\_single\_base & NaN & 0.258000 & NaN & 0.000000 \\
\end{longtable}


\subsection{Internal consistency checks}
Key identities satisfied by the generated outputs include:
\begin{itemize}
  \item \textbf{Component identity:} \(\texttt{direct\_tax} = \texttt{tax\_gasoline}+\texttt{tax\_oil}+\texttt{tax\_ng}\).
  \item \textbf{Wedge identity:} \(\texttt{gross\_cost} = \texttt{direct\_tax}(1+\texttt{kappa})\) and \(\texttt{indirect\_cost} = \texttt{gross\_cost}-\texttt{direct\_tax}\).
  \item \textbf{Net identity:} \(\texttt{net\_type} = \texttt{gross\_cost}-\texttt{CCR\_type}\) for each household type and rural variant.
  \item \textbf{2025 cessation convention:} 2025 statutory rates are set to zero in the script, implying \(\texttt{direct\_tax}=\texttt{gross\_cost}=0\) (and rebates also set to zero for 2025).
\end{itemize}

\section{Outputs and reproducibility}
Running \texttt{compute\_ggppa\_progressivity\_v4.py} in the same directory as the input \texttt{.xlsx} files produces:
\begin{itemize}
  \item \texttt{ggppa\_progressivity\_results\_v4.csv}: main province$\times$quintile$\times$year incidence outputs (with intermediate calculations),
  \item \texttt{ggppa\_kakwani\_indices\_v4.csv}: Kakwani indices for selected measures,
  \item \texttt{ggppa\_finance\_crosswalk\_v4.csv}: indirect wedge calibration diagnostic.
\end{itemize}
A full variable dictionary is provided in \texttt{DATA\_DICTIONARY.md}.

\section*{References (APA 7)}
Canada Revenue Agency. (n.d.). \textit{Fuel charge rates}. Government of Canada. Retrieved February 10, 2026, from \url{https://www.canada.ca/en/revenue-agency/services/forms-publications/publications/fcrates/fuel-charge-rates.html}

Canada, Department of Finance. (2018). \textit{Backgrounder: Minister of Finance confirms amounts of Climate Action Incentive payments for 2019}. Government of Canada. Retrieved February 10, 2026, from \url{https://www.canada.ca/en/department-finance/news/2018/12/minister-of-finance-confirms-amounts-of-climate-action-incentive-payments-for-2019.html}

Canada, Department of Finance. (2019). \textit{Climate Action Incentive payment amounts for 2020}. Government of Canada. Retrieved February 10, 2026, from \url{https://www.canada.ca/en/department-finance/news/2019/12/climate-action-incentive-payment-amounts-for-2020.html}

Canada, Department of Finance. (2020). \textit{Climate Action Incentive payment amounts for 2021}. Government of Canada. Retrieved February 10, 2026, from \url{https://www.canada.ca/en/department-finance/news/2020/12/climate-action-incentive-payment-amounts-for-2021.html}

Canada, Department of Finance. (2022a). \textit{Climate Action Incentive payment amounts for 2022--23}. Government of Canada. Retrieved February 10, 2026, from \url{https://www.canada.ca/en/department-finance/news/2022/03/climate-action-incentive-payment-amounts-for-2022-23.html}

Canada, Department of Finance. (2022b). \textit{Climate Action Incentive payment amounts for 2023--24}. Government of Canada. Retrieved February 10, 2026, from \url{https://www.canada.ca/en/department-finance/news/2022/11/climate-action-incentive-payment-amounts-for-2023-24.html}

Canada, Department of Finance. (2024). \textit{Canada Carbon Rebate amounts for 2024--25}. Government of Canada. Retrieved February 10, 2026, from \url{https://www.canada.ca/en/department-finance/news/2024/02/canada-carbon-rebate-amounts-for-2024-25.html}

Canada, Environment and Climate Change Canada. (2021). \textit{Greenhouse Gas Pollution Pricing Act: Annual report for 2021}. Government of Canada. Retrieved February 10, 2026, from \url{https://www.canada.ca/en/environment-climate-change/services/climate-change/pricing-pollution-how-it-will-work/greenhouse-gas-annual-report-2021.html}

Canada, Environment and Climate Change Canada. (2023). \textit{Greenhouse Gas Pollution Pricing Act 2023}. Government of Canada. Retrieved February 10, 2026, from \url{https://www.canada.ca/en/environment-climate-change/services/climate-change/pricing-pollution-how-it-will-work/greenhouse-gas-annual-report-2023.html}

Office of the Parliamentary Budget Officer. (2022). \textit{A distributional analysis of federal carbon pricing under A Healthy Environment and A Healthy Economy}. Retrieved February 10, 2026, from \url{https://www.pbo-dpb.ca/en/publications/RP-2122-032-S--distributional-analysis-federal-carbon-pricing-under-healthy-environment-healthy-economy--une-analyse-distributive-tarification-federale-carbone-dans-cadre-plan-un-environnement-sain-une-eco}

Statistics Canada. (n.d.). \textit{Table 11-10-0223-01: Household spending, Canada, regions and provinces}. Retrieved February 10, 2026.

Statistics Canada. (n.d.). \textit{Table 18-10-0001-01: Consumer Price Index, annual average, not seasonally adjusted}. Retrieved February 10, 2026.

Statistics Canada. (n.d.). \textit{Table 36-10-0101-01: Number of households, by household income quintile}. Retrieved February 10, 2026.

\end{document}
